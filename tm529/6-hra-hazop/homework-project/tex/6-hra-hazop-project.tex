\documentclass[11pt,a4paper]{article}
\usepackage[lmargin=1in,rmargin=1in,tmargin=1in,bmargin=1in]{geometry}
\usepackage[pagewise]{lineno} %line numbering
\usepackage{setspace}
\usepackage{ulem} %strikethrough - do not \sout{\cite{}}
\usepackage{xcolor} %change font color
\usepackage{graphicx}
\usepackage{filecontents}
\usepackage{tablefootnote}
\usepackage{footnotehyper}
\usepackage{subfig}
\usepackage[yyyymmdd]{datetime} %date format
\renewcommand{\dateseparator}{.}
\graphicspath{{../img/}} %path to graphics
\setcounter{secnumdepth}{5} %set subsection to nth level
\usepackage{caption}
\captionsetup[table]{skip=11pt} %sets a space after table caption
\usepackage{times}
\usepackage{tabto} %general tabbed spacing
\usepackage{longtable} %need to put label at top under caption then \\ - use spacing
\usepackage[stable,hang,flushmargin]{footmisc} %footnotes in section titles and no indent
\usepackage[round]{natbib} %parenthesis instead of brackets for inline citations
\usepackage{enumitem}
\usepackage{boldline}
\usepackage{makecell}
\usepackage{booktabs}
\usepackage{amssymb}
\usepackage{amsmath}
\usepackage{physics}
\usepackage{tabularx}
\usepackage{multirow}
\usepackage{lscape}
\usepackage{array}
\usepackage{caption}
\usepackage[labelfont=bf]{caption}
\usepackage{chngcntr}
\usepackage{hyperref}

\newcommand{\edit}[1]{\textcolor{blue}{#1}} %shortcut for changing font color on revised text
\newcommand{\fn}[1]{\footnote{#1}} %shortcut for footnote tag
\newcommand*\sq{\mathbin{\vcenter{\hbox{\rule{.3ex}{.3ex}}}}} %makes a small square as a separator $\sq$
\renewcommand\labelenumi{(\theenumi)} %changes 1. to (1) in enumerated list

\usepackage{fancyhdr}
\pagestyle{fancy}
\fancyhf{} %move page number to bottom right
\renewcommand{\headrulewidth}{0.5pt} %turn off line in header
\lhead{\scriptsize TM529 - Risk assessment}
\chead{\scriptsize \today}
\rhead{\scriptsize Project 6 - HRA and HAZOP}
\rfoot{\thepage}

\begin{filecontents}{references.bib}
    @misc{
        ,
        author = {{}},
        title = {{}},
        year = {}
    }
    @article{
        ,
        author = {{}},
        journal = {},
        pages = {},
        title = {{}},
        volume = {},
        year = {}
    }
    @techreport{
        ,
        author = {{}},
        title = {{}},
        year = {},
        institution = {},
        number = {}
    }
\end{filecontents}

\begin{document}

\begin{titlepage}
    \title{
        TM529 - Risk assessment\\
        Project 6 - HRA and HAZOP\\
    }
    \author{
        Name
        \\ \\ \\
        University of Idaho $\sq$ Idaho Falls Center for Higher Education
        \\ \\
        Nuclear Engineering and Industrial Management Department
        \\ \\ \\
        email 
    }
\clearpage %not have page number on title page
\maketitle
\vspace*{\fill}
\begin{flushright}{
        Total - 300
}
\end{flushright}
\thispagestyle{empty} %start with page number 1 on second page
\end{titlepage}

\begin{enumerate}[leftmargin=*,topsep=0pt,font=\bfseries]
    \item\textbf{(50) There was yet another catastrophic airline failure in 2005 on \href{https://en.wikipedia.org/wiki/Helios_Airways_Flight_522}{Helios Airways}. There is also a video in the case studies chapter of the \href{https://courses.lumenlearning.com/uidaho-riskassessment/chapter/contemporary-cases-in-risk-assessment-2/}. Briefly explain the accident. Comment on how this (potentially) was a failure of human reliability analysis and how HRA could have been applied to prevent this accident. Consider this to be at the same level of content as the news pieces.}
    \item[] See - \href{https://www.flightglobal.com/news/articles/investigation-dispels-myths-around-helios-airways-cr-209984/}{Investigation dispels myths around Helios Airways crash}
        \vspace{\baselineskip}

        
        
        
        
        
        
        
        
        
        
        
        
        
        
        
        
        
        
        
        
        
        
        
        
        \newpage
    \item\textbf{(100) Since there have been several PHAs and FMEAs conducted already, continue your risk assessment and perform a limited HAZOP on one of the systems of interest or something new. Perform the HAZOP for a total of 6 deviations; e.g., a single component, 2 parameters with 3 guidewords; 3 parameters with 2 guidewords, or a similar combination for 2 components. Construct a HAZOP table and sufficiently explain your reasoning.}
        \vspace{\baselineskip}

        
        
        
        
        
        
        
        
        
        
        
        
        
        
        
        
        
        
        
        
        
        
        
        
        
        
        
        
        
        
        
        
        
        
        
        \newpage
    \item\textbf{(100) You are driving on I-84, west of Twin Falls, heading to Boise for some ridiculous symposium. It is nighttime and late in March. Conduct a HAZOP for the commute (i.e., operation of the car) as follows and sufficiently explain your reasoning.
        \begin{itemize}
            \item Since the car would be the component, generate 5 operational parameters that affect the commute.
            \item Generate 2 or 3 associated guidewords for each parameter.
            \item Briefly explain the resulting operational deviations (deviation = guideword + parameter)
            \item Again, select a combination of 6 deviations and complete the HAZOP table.
        \end{itemize}}
        \vspace{\baselineskip}


        
        
        
        
        
        
        
        
        
        
        
        
        
        
        
        
        
        
        
        
        
        
        
        
        
        
        
        
        
        
        
        
        
        
        
        
        
        
        \newpage
    \item\textbf{(50) Watch the PBS news piece on the \href{https://youtu.be/Quh6fX57YxY}{Oroville Dam}. Is this a failure of risk assessment or management? Should the events leading to the flooding be included in the design basis or beyond design basis?}
        \vspace{\baselineskip}


























        \newpage
    \item\textbf{(100) Since you have identified initiating events already from the PHA and FMEA, take the two initiating events with the highest risk and construct and event tree for each. Please discuss your reasoning and the sequence from initiating event to final state.  Try to assign nominal frequencies to the events and discuss the risk of the final states.}
        \vspace{\baselineskip}
        
        
        
        
        
        
        
        
        
        
        
        
        
        
        
        
        
        
        
        
        
        
        
        
        
        
        
        
        
        \newpage
    \item\textbf{(100) In 1985, Japanese Airlines flight 123 crashed into a mountain shortly after take off. It seems that the initiating event was fatigue failure of the aft bulkhead due to errors in repairs. Briefly explain what happened. Construct an event tree similarly. Start with the \href{http://www.shippai.org/fkd/en/cfen/CB1071008.html}{Failure Knowledge Database} and the \href{https://lessonslearned.faa.gov/ll_adsearch_results.cfm?TabID=5}{USA FAA}. There is actually a lot of information out there about this case. Estimate nominal frequencies for the events.}
        \vspace{\baselineskip}
        
        
        
        
        
        
        
        
        
        
        
        
        
        
        
        
        
        
        
        
        
        
        
        
        
        
        
        
        
        
        
        
        
        
        
        \newpage
    \item\textbf{(100) A fire is a common failure event anywhere, and any risk assessment will also include analysis for fire protection. \href{https://images.sampletemplates.com/wp-content/uploads/2016/06/02114906/Fault-Tree-Analysis-Format.jpg}{A fire requires fuel AND oxygen AND ignition}. Based on the given fault tree, present a fault tree analysis. Select any fire incident. The basic events need not be exactly the seven shown here; there could be more or less, and any combination of AND or OR gates may be needed. However, they all will lead to the three intermediate events in some way. Try to formulate nominal probabilities for the basic events and compute a final probability for the top level event; i.e., the fire. Finally, provide some analysis on the probability of the fire in terms of risk acceptability, and discuss any potential mitigation strategies. Anyone feeling adventurous can try any open source FTA software (\textit{not required}).}
        \vspace{\baselineskip}
        
        
        
        
        
        
        
        
        
        
        
        
        
        
        
        
        
        
        
        
        
        
        
        
        
        
        
        
        
        
        
        
        
        
        \newpage
    \item\textbf{(100) Select two of the events in the sequence from the event tree and construct a fault tree for each. Discuss how and why the fault tree was constructed and try to establish a frequency for the event based on basic events in the fault tree.}
        \vspace{\baselineskip}





















































\end{enumerate}

\newpage 

\bibliographystyle{ieeetr}
\setlength{\bibhang}{0pt}
\bibliography{references}

\end{document}
