%@TheDoctorRAB
%Standard white paper/preproposal format when there is a page limit
%neup.bst designed to use numbered citations in order of appearance, short author list
\documentclass[11pt,letterpaper]{article}
\usepackage[lmargin=1in,rmargin=1in,tmargin=1in,bmargin=1in]{geometry}
\usepackage[pagewise]{lineno} %line numbering
\usepackage{setspace}
\usepackage{ulem} %strikethrough
\usepackage{xcolor,colortbl} %change font color
\usepackage{graphicx}
\usepackage{filecontents}
\usepackage{tablefootnote}
\usepackage{subfig}
\usepackage[yyyymmdd]{datetime}
\renewcommand{\dateseparator}{.}
\graphicspath{{../img/}}
\setcounter{secnumdepth}{5} %set subsection to nth level
\usepackage{times}
\usepackage{enumitem}
\usepackage{multirow}
\usepackage{physics}
\usepackage{amssymb}
\usepackage{amsmath}
\usepackage{float}
\usepackage{longtable} %need to put label at top under caption then \\ - use spacing
%\usepackage[labelfont=bf]{caption}

\usepackage[singlelinecheck=false]{caption}
\captionsetup[table]{skip=0pt} %sets a space after table caption
\captionsetup[figure]{skip=0pt,labelformat={default},labelsep=period,name={Fig.}} %sets space above caption, 'figure' format

\usepackage{wrapfig}
\setlength{\intextsep}{0.10mm}
\setlength{\columnsep}{0.10mm}

\usepackage[numbers,sort&compress]{natbib} %use 'numbers' for numbered citations; 'round' for () instead [] for inline citations
%\setlength{\bibsep}{0pt} %sets space between references
\renewcommand{\bibsection}{} %suppresses large 'references' heading
\renewcommand\bibpreamble{\vspace{-0.2\baselineskip}} %sets spacing after heading if not using default references heading

\newcommand{\edit}[1]{\textcolor{blue}{#1}} %shortcut for changing font color on revised text
\newcommand{\fn}[1]{\footnote{#1}} %shortcut for footnote tag
\newcommand*\sq{\mathbin{\vcenter{\hbox{\rule{.3ex}{.3ex}}}}} %makes a small square as a separator $\sq$
\newcommand{\x}{\cellcolor{lightgray}} %use to shade in table cell

\newcolumntype{L}[1]{>{\raggedright\let\newline\\\arraybackslash\hspace{0pt}}p{#1}} %uses \raggedright with p{} in table column

%\paragraph and \subparagraph modifiers
% no indent for each
% after \z@ pads whitespace on the top
% em sets the distance after heading to text horizonally

\makeatletter
\renewcommand\paragraph{%
    \@startsection{paragraph}{4}{\z@ }{0.55\baselineskip}{-1em}
    {\normalfont \normalsize \bfseries}}%

\makeatletter
\renewcommand\subparagraph{%
    \@startsection{subparagraph}{5}{\z@ }{0.45\baselineskip}{-1em}
    {\normalfont \normalsize \itshape }}%

\makeatletter
\renewcommand\subsection{%
    \@startsection{subsection}{2}{\z@ }{0.75\baselineskip}{0.25\baselineskip}
    { \large \bfseries}}%

\usepackage{fancyhdr}
\pagestyle{fancy}
\fancyhf{} %move page number to bottom right
\renewcommand{\headrulewidth}{0.5pt} %turn off line in header
\lhead{\scriptsize Name}
\chead{\scriptsize TM529 - Project white paper title}
\rhead{\scriptsize \today}
\rfoot{\thepage}

\begin{document}

{\centering 
    \textbf{Title\\
    TM529 - Risk Assessment\\
    }
    Name - University of Idaho $\sq$ Idaho Falls Center for Higher Education\\
    Department of Nuclear Engineering and Industrial Management
\par
}

\paragraph*{Summary of the Proposed Project.} 
\noindent

\paragraph*{Motivation.}
\noindent

\paragraph*{Overall Objective.}
\noindent

\paragraph*{Importance and Relevance to Objectives.}
\noindent

\paragraph*{Logical Path, Work Scope, Description of Tasks.} The Logical Path and the proposed Work Scope is described below in terms of the defined Tasks for the project.

\paragraph*{Workscope 1. Title.}
\subparagraph*{Task I. Title.}
\noindent

\subparagraph*{Task II. Title.}
\noindent

\paragraph*{Workscope 2. Title.}
\subparagraph*{Task III. Title.}
\noindent

\subparagraph*{Task N. Title.}
\noindent

\vspace{0.50\baselineskip}

\begin{table}[h!]
    \centering
    \caption*{\textbf{Timeframe for Execution of Proposed Project. Schedule, Roles, and Responsibilities.}}
    \begin{tabular}{|l|c|c|c|c|c|}
        \hline
        \multicolumn{1}{|c|}{\multirow{2}{*}{\textbf{TASKS}}}& 
        \multicolumn{5}{|c|}{\textbf{Semester months}}\\
        \cline{2-6}
        &
        \textbf{J}& 
        \textbf{F}& 
        \textbf{M}& 
        \textbf{A}& 
        \textbf{M}
        \\
        \hline
        I. Title
        &\x\textbf{X} 
        & 
        & 
        & 
        & 
        \\
        \hline
        II. Title
        &\x\textbf{X} 
        & 
        & 
        & 
        & 
        \\
        \hline
        III. Title
        &\x\textbf{X} 
        & 
        & 
        & 
        & 
        \\
        \hline
        IV. Title
        &\x\textbf{X} 
        & 
        & 
        & 
        &
        \\
        \hline
        V. Title
        &\x\textbf{X} 
        & 
        & 
        & 
        &  
        \\
        \hline
        VI. Title
        & 
        & 
        & 
        & 
        &\x\textbf{X} 
        \\
        \hline
        VII. Title
        & 
        & 
        & 
        & 
        &\x\textbf{X}
        \\
        \hline
    \end{tabular}
    \label{tab-timeframe}
\end{table}

\newpage

\noindent\textbf{References}
\bibliographystyle{neup}
\setlength{\bibhang}{0pt}
\bibliography{references}

%wrap figure around text
%move to wherever in the text 
%{l} fixes it on the left margin
%\begin{wrapfigure}{l}{0.x\textwidth}
%        \includegraphics[width=0.(x-2)\textwidth]{}
%    \caption{}
%    \label{fig-}
%\end{wrapfigure}

\end{document}
