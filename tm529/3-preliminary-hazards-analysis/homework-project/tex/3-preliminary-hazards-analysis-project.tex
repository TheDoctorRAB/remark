\documentclass[11pt,a4paper]{article}
\usepackage[lmargin=1in,rmargin=1in,tmargin=1in,bmargin=1in]{geometry}
\usepackage[pagewise]{lineno} %line numbering
\usepackage{setspace}
\usepackage{ulem} %strikethrough - do not \sout{\cite{}}
\usepackage{xcolor} %change font color
\usepackage{graphicx}
\usepackage{filecontents}
\usepackage{tablefootnote}
\usepackage{footnotehyper}
\usepackage{subfig}
\usepackage[yyyymmdd]{datetime} %date format
\renewcommand{\dateseparator}{.}
\graphicspath{{../img/}} %path to graphics
\setcounter{secnumdepth}{5} %set subsection to nth level
\usepackage{caption}
\captionsetup[table]{skip=11pt} %sets a space after table caption
\usepackage{times}
\usepackage{tabto} %general tabbed spacing
\usepackage{longtable} %need to put label at top under caption then \\ - use spacing
\usepackage[stable,hang,flushmargin]{footmisc} %footnotes in section titles and no indent
\usepackage[round]{natbib} %parenthesis instead of brackets for inline citations
\usepackage{enumitem}
\usepackage{boldline}
\usepackage{makecell}
\usepackage{booktabs}
\usepackage{amssymb}
\usepackage{amsmath}
\usepackage{physics}
\usepackage{tabularx}
\usepackage{multirow}
\usepackage{lscape}
\usepackage{array}
\usepackage{caption}
\usepackage[labelfont=bf]{caption}
\usepackage{chngcntr}
\usepackage{hyperref}

\newcommand{\edit}[1]{\textcolor{blue}{#1}} %shortcut for changing font color on revised text
\newcommand{\fn}[1]{\footnote{#1}} %shortcut for footnote tag
\newcommand*\sq{\mathbin{\vcenter{\hbox{\rule{.3ex}{.3ex}}}}} %makes a small square as a separator $\sq$
\renewcommand\labelenumi{(\theenumi)} %changes 1. to (1) in enumerated list

\usepackage{fancyhdr}
\pagestyle{fancy}
\fancyhf{} %move page number to bottom right
\renewcommand{\headrulewidth}{0.5pt} %turn off line in header
\lhead{\scriptsize TM529 - Risk assessment}
\chead{\scriptsize \today}
\rhead{\scriptsize Project 3 - Preliminary hazards analysis}
\rfoot{\thepage}

\begin{filecontents}{references.bib}
    @misc{
        ,
        author = {{}},
        title = {{}},
        year = {}
    }
    @article{
        ,
        author = {{}},
        journal = {},
        pages = {},
        title = {{}},
        volume = {},
        year = {}
    }
    @techreport{
        ,
        author = {{}},
        title = {{}},
        year = {},
        institution = {},
        number = {}
    }
\end{filecontents}

\begin{document}

\begin{titlepage}
    \title{
        TM529 - Risk assessment\\
        Project 3 - Preliminary hazards analysis\\
    }
    \author{
        Name
        \\ \\ \\
        University of Idaho $\sq$ Idaho Falls Center for Higher Education
        \\ \\
        Nuclear Engineering and Industrial Management Department
        \\ \\ \\
        email 
    }
\clearpage %not have page number on title page
\maketitle
\vspace*{\fill}
\begin{flushright}{
        Total - 300
}
\end{flushright}
\thispagestyle{empty} %start with page number 1 on second page
\end{titlepage}

\begin{enumerate}[leftmargin=*,topsep=0pt,font=\bfseries]
    \item\textbf{(100) Conduct a formalized PHA from the topic selected in Homework Project 1, problem 1. Or pick a new topic. Tables 6.1 and 6.2 in the book are a good place to start, but feel free to work in whatever way is most comfortable. Apply some type of visual ranking tool, whether a risk matrix, from Chapter 3 slides, generalized Farmer’s chart, something else that is out there. Suggest design criteria. Also, identify what agency would be involved for relevant regulations.}
        \vspace{\baselineskip}
    
        
        
        
        
        
        
        
        
        
        
        
        
        
        \newpage
    \item\textbf{(100) Conduct a similar PHA for your own workspace and a common area. (\textit{Take pics!}). Evaluate the workspace and how potential hazards can be mitigated.}
        \vspace{\baselineskip}
        
        
        
        
        
        
        
        
        
        
        
        
        
        
        \newpage
    \item\textbf{(100) Can system risk be properly evaluated by means of a PHA? (Short essay).}
        \vspace{\baselineskip}

        
        
        
        
        
        
        
        
        
        
        
        
        

        
        
        \newpage
    \item\textbf{(100) Watch the piece about \href{http://www.pbs.org/newshour/bb/north-dakotas-oil-fields-deadly-workers/}{Why North Dakota’s oil fields are so deadly for workers}. Please discuss this topic in the same manner as the Oklahoma fracking piece. \href{https://youtu.be/jYusNNldesc}{John Oliver} also put together a similar piece, for further interest (\textit{NSFW}).}
        \vspace{\baselineskip}


































\end{enumerate}

\newpage 

\bibliographystyle{ieeetr}
\setlength{\bibhang}{0pt}
\bibliography{references}

\end{document}
