\documentclass[11pt,a4paper]{article}
\usepackage[lmargin=1in,rmargin=1in,tmargin=1in,bmargin=1in]{geometry}
\usepackage[pagewise]{lineno} %line numbering
\usepackage{setspace}
\usepackage{ulem} %strikethrough - do not \sout{\cite{}}
\usepackage{xcolor} %change font color
\usepackage{graphicx}
\usepackage{filecontents}
\usepackage{tablefootnote}
\usepackage{footnotehyper}
\usepackage{subfig}
\usepackage[yyyymmdd]{datetime} %date format
\renewcommand{\dateseparator}{.}
\graphicspath{{../img/}} %path to graphics
\setcounter{secnumdepth}{5} %set subsection to nth level
\usepackage{caption}
\captionsetup[table]{skip=11pt} %sets a space after table caption
\usepackage{times}
\usepackage{tabto} %general tabbed spacing
\usepackage{longtable} %need to put label at top under caption then \\ - use spacing
\usepackage[stable,hang,flushmargin]{footmisc} %footnotes in section titles and no indent
\usepackage[round]{natbib} %parenthesis instead of brackets for inline citations
\usepackage{enumitem}
\usepackage{boldline}
\usepackage{makecell}
\usepackage{booktabs}
\usepackage{amssymb}
\usepackage{amsmath}
\usepackage{physics}
\usepackage{tabularx}
\usepackage{multirow}
\usepackage{lscape}
\usepackage{array}
\usepackage{caption}
\usepackage[labelfont=bf]{caption}
\usepackage{chngcntr}
\usepackage{hyperref}

\newcommand{\edit}[1]{\textcolor{blue}{#1}} %shortcut for changing font color on revised text
\newcommand{\fn}[1]{\footnote{#1}} %shortcut for footnote tag
\newcommand*\sq{\mathbin{\vcenter{\hbox{\rule{.3ex}{.3ex}}}}} %makes a small square as a separator $\sq$
\renewcommand\labelenumi{(\theenumi)} %changes 1. to (1) in enumerated list

\usepackage{fancyhdr}
\pagestyle{fancy}
\fancyhf{} %move page number to bottom right
\renewcommand{\headrulewidth}{0.5pt} %turn off line in header
\lhead{\scriptsize NE450 - Principles of nuclear engineering}
\chead{\scriptsize \today}
\rhead{\scriptsize Project 5 - Monte Carlo methods and MCNP}
\rfoot{\thepage}

\begin{filecontents}{references.bib}
    @misc{
        ,
        author = {{}},
        title = {{}},
        year = {}
    }
    @article{
        ,
        author = {{}},
        journal = {},
        pages = {},
        title = {{}},
        volume = {},
        year = {}
    }
    @techreport{
        ,
        author = {{}},
        title = {{}},
        year = {},
        institution = {},
        number = {}
    }
\end{filecontents}

\begin{document}

\begin{titlepage}
    \title{
        NE450 - Principles of nuclear engineering\\
        Project 5 - Monte Carlo methods and MCNP\\
    }
    \author{
        Name
        \\ \\ \\
        University of Idaho $\sq$ Idaho Falls Center for Higher Education
        \\ \\
        Nuclear Engineering and Industrial Management Department
        \\ \\ \\
        email 
    }
\clearpage %not have page number on title page
\maketitle
\vspace*{\fill}
\begin{flushright}{
        Total - 450 
}
\end{flushright}
\thispagestyle{empty} %start with page number 1 on second page
\end{titlepage}

\noindent For problems 1 - 3 use Monte Carlo techniques to obtain the solutions. \\

\begin{enumerate}[leftmargin=*,topsep=0pt,font=\bfseries]
    \item\textbf{(30) Solve for G using Monte Carlo techniques. Solve the integral analytically and graph g(x). Also plot G v N for $N = 10,\;10^2.\;10^3,\;10^4,\;10^5,\;10^6$.}
        \begin{equation}
            G = \int_0^1 g(x)dx 
        \end{equation}
        \begin{equation}
            g(x)=1-e^{-x}
        \end{equation}
        \vspace{0.25in}\\

        
        
        
        
        
        
        
        
        
        
        
        
        
        
        \newpage 
    \item\textbf{(30) Do the same for the following function. This g(x) does not have an analytical solution. However, you can use another numerical solver to compare the Monte Carlo result.}
        \begin{equation}
            \int_0^{\frac{\pi}{2}}sin(x^2)dx
        \end{equation}
        \vspace{0.25in}\\














        \newpage 
    \item\textbf{(30) Approximate $\sqrt{2}$ in a similar manner to the way we approximated $\pi$.}
        \vspace{0.25in}\\


















        \newpage 
    \item\textbf{(50) Conduct a short modeling study of the metal fuel alloy for the hot cell facility using - }
        \begin{itemize}[leftmargin=*,topsep=0pt]
            \item\href{https://github.com/TheDoctorRAB/mcnpx.decks/blob/master/neutron.flux/input/4_ff.alloy.inp}{4\_ff.alloy.inp (flux)}
            \item\href{https://github.com/TheDoctorRAB/mcnpx.decks/blob/master/neutron.flux/input/4d_ff.alloy.inp}{4d\_ff.alloy.inp (dose)}
        \end{itemize}
    \item[]See also the related \href{https://piazza.com/class_profile/get_resource/kbjsrzm65to4ul/kbjss2sfmbe4wu}{paper} in piazza for more information.
    \item[]Apply the following procedure - 
        \begin{itemize}
            \item Look at the original geometry in the plotter/VisEd.
            \item Modify the facility to only include the SE and SW cells.
            \item Use MCNP to compute the volume averaged (F4) flux tallies for the SE cell and SW cell.
            \item Use a neutron emission rate of of $1.1 \times 10^7 \; n/s/g$ for 24 grams of material. 
            \item Increase NPS from the original files to reduce standard error.
            \item Start with a wall thickness of 15 cm using the material already included in the deck. It is a form of borated concrete that is common to these kinds of facilities. Increase the wall thickness until the dose rate falls below $1 \mu Sv/h$ and the relative flux falls below 0.01.
            \item Plot dose rate v wall thickness and the relative flux v wall thickness.
            \item Justify that the results are scientifically sound.
        \end{itemize}
    \item[]Is this wall thickness reasonable? As in, could a facility be practically built like this using current engineering design techniques?
    \item[]Include the MCNP file at the end in an appendix.
        \vspace{0.25in}\\






















        \newpage 
    \item[] For the criticality models, to get full credit - 
        \begin{itemize}[leftmargin=*,topsep=0pt]
            \item Include a screenshot of the model from the VisEd/plotter.
            \item Use finite geometries.
            \item Design geometries that will minimize leakage. Show (as part of making the mcnp file and results; not calculating by hand) that leakage has been minimized.
            \item For criticality, try to get to 3 9s or 0s (.999x, 1.000x) for the mean, and 68\% confidence. Bonus for 4 9s/0s.
            \item Report output in a table - k, standard deviation, 68\% confidence, 95\% confidence, and 99\% confidence.
            \item Justify the results are scientifically sound.
            \item Include the input deck in the appendix.
        \end{itemize}
    \item[]\textbf{PROTIP - }k can vary weird when your trying to get the critical radius to 4 or 5 decimal places. Study the KCODE parameters. You could also add more particles on KSRC, but be careful where you place them.
        \newpage
    \item\textbf{(20) What is the critical mass of a bare sphere of plutonium containing (1) $95.5\% \; ^{239}Pu$ and (2) $80\% \; ^{239}Pu$, where the rest is $^{238}Pu$?}
        \vspace{0.25in}\\




















        \newpage 
    \item\textbf{(20) What is the critical mass for the above, but with a thin nickel shell of 0.10 cm?}
        \vspace{0.25in}\\
















        \newpage 
    \item\textbf{(20) What is the critical mass of pure $^{239}Pu$ of a bare cylinder?}
        \vspace{0.25in}\\























        \newpage 
    \item\textbf{(30) Taking the bare sphere $^{239}Pu$ model, what is the optimal reflector that minimizes the critical mass?}
    \item\textbf{(30) Do the same for $^{235}U$.}
    \item\textbf{(30) Do the same for $^{233}U$.}
    \item[] For these next three problems, select 3 to 5 typical reflector materials for each fissionable source. Make a table for each source with results from the reflectors. 






























        \newpage 
    \item\textbf{(20) Put all the results from the reflector problems together in a table. Which reflector material is minimal and why, neutronically speaking?}
        \vspace{0.25in}\\

































        \newpage 
    \item\textbf{(50) Three unreflected aluminum cylinders contain $U(93.2)O_2F_2$ water solutions. The inside cylinder diameter and critical height measured 20.3 cm and 41.4 cm. The aluminum container had a density of $2.71 \; g/cm^3$ and was 0.15 cm thick. The three cylinders were set in an equilateral configuration with a surface separation of 0.38 cm. The solution concentration parameters were $0.90 \; g(^{235}U)/cm^3$ with $H:^{235}U = 309$}.
    \item[]\textbf{It was estimated that the solution density was approximately $1.131 \; g/cm^3$ and consisted of $0.0021345\;^{235}U, \; 0.00015382\;^{238}U, \; 0.33383\;O, \; 0.65930\;H, \; 0.0045756\;F \; atoms/b-cm$.}
    \item[]\textbf{MCNP gives $k = 0.9991\pm0.0011$. Get within 15\% for full credit.} 
    \item[]\textbf{Reproduce the model to get the result. See the \href{https://courses.lumenlearning.com/uidaho-nuclear/chapter/mcnp/}{MCNP benchmark document} in the OER for guidance.}
        \vspace{0.25in}\\
























        \newpage 
    \item\textbf{(20) Find the critical mass for a bare cylinder of 10.9\% enriched U with a density of 18.63 g/cc.}
        \vspace{0.25in}\\
































        \newpage 
    \item\textbf{(20) Find the minimum critical mass for an infinite graphite reflected 93.5\% enriched U sphere. Use 18.8 g/cc for the U density and just use carbon for the graphite.}
        \vspace{0.25in}\\





























        \newpage 
    \item\textbf{(20) Find the critical mass of 97.67\% enriched U cube in an infinite water reflector. Use a density of 18.794 g/cc}.
        \vspace{0.25in}\\






















\end{enumerate}

\newpage

\noindent{\Large\textbf{Hot cell MCNP input decks}}

\newpage

\noindent{\Large\textbf{Criticality MCNP input decks}}

\newpage

\bibliographystyle{ieeetr}
\setlength{\bibhang}{0pt}
\bibliography{references}

\end{document}
