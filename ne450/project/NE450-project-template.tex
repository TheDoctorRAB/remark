\documentclass[11pt,a4paper]{article}
\usepackage[lmargin=1in,rmargin=1in,tmargin=1in,bmargin=1in]{geometry}
\usepackage[pagewise]{lineno} %line numbering
\usepackage{setspace}
\usepackage{ulem} %strikethrough - do not \sout{\cite{}}
\usepackage{xcolor} %change font color
\usepackage{graphicx}
\usepackage{filecontents}
\usepackage{tablefootnote}
\usepackage{footnotehyper}
%\usepackage{subfig}
\usepackage[yyyymmdd]{datetime} %date format
\renewcommand{\dateseparator}{.}
\graphicspath{{../img/}} %path to graphics
\setcounter{secnumdepth}{5} %set subsection to nth level
\usepackage{caption}
\captionsetup[table]{skip=11pt} %sets a space after table caption
\usepackage{times}
\usepackage{tabto} %general tabbed spacing
\usepackage{longtable} %need to put label at top under caption then \\ - use spacing
\usepackage[stable,hang,flushmargin]{footmisc} %footnotes in section titles and no indent
\usepackage[round]{natbib} %parenthesis instead of brackets for inline citations
\usepackage{enumitem}
\usepackage{boldline}
\usepackage{makecell}
\usepackage{booktabs}
\usepackage{amssymb}
\usepackage{amsmath}
\usepackage{physics}
\usepackage{tabularx}
\usepackage{multirow}
\usepackage{lscape}
\usepackage{array}
\usepackage{caption}
\usepackage{subcaption}
\usepackage[labelfont=bf]{caption}
\usepackage{chngcntr}

%\counterwithin{table}{section}

%\usepackage{xr}
%\externaldocument{} %aux file needed

\newcommand{\edit}[1]{\textcolor{blue}{#1}} %shortcut for changing font color on revised text
\newcommand{\fn}[1]{\footnote{#1}} %shortcut for footnote tag
\newcommand*\sq{\mathbin{\vcenter{\hbox{\rule{.3ex}{.3ex}}}}} %makes a small square as a separator $\sq$
\newcommand{\x}{\cellcolor{lightgray}} %use to shade in table cell

\usepackage{fancyhdr}
\pagestyle{fancy}
\fancyhf{} %move page number to bottom right
\renewcommand{\headrulewidth}{0.5pt}
\lhead{\scriptsize Name}
\chead{\scriptsize NE450 - Project report title}
\rhead{\scriptsize \today}
\rfoot{\thepage}

\begin{filecontents}{references.bib}
    @article{
        xxxxx,
        author = {},
        journal = {},
        pages = {},
        title = {{}},
        volume = {},
        year = {}
    }
    @conference{
        yyyyy,
        author = {},
        title = {{}},
        year = {},
        organization = {},
        address = {}
    }
    @techreport{
        zzzzz,
        author = {},
        title = {{}},
        year = {},
        institution = {},
        number = {},
        type = {~}
    }
    @book{
        nnnnn,
        author = {},
        title = {{}},
        publisher = {},
        year = {},
        isbn = {}
    }
\end{filecontents}

\begin{document}

\begin{titlepage}
    \title{Title}
    \author{
        Name \\
        NE450: Principles of Nuclear Engineering Course Design Project
        \\ \\
        University of Idaho $\sq$ Idaho Falls Center for Higher Education\\[0.05in]
        Engineering/Technology Management, Industrial Technology\\and\\Nuclear Engineering Department
        \\ \\
        email
    }
\clearpage % not have page number on title page
\maketitle
\vspace*{\fill}
\begin{flushright}{
        \noindent Number of pages - XX \\
        \noindent Number of tables - XX \\
        \noindent Number of figures - XX
}
\end{flushright}
\thispagestyle{empty} % start with page number 1 on second page
\end{titlepage}

\onehalfspacing
\linenumbers
\pagewiselinenumbers
\modulolinenumbers[3] % line numbering interval

\noindent\Large{\textbf{Executive Summary}} \label{exec-summ} \\

\newpage

\section{Introduction} \label{introduction}
Expand on the white paper in this section to introduce the project.

\subsection{Motivation}

\subsection{Goals}

\newpage

\section{Background} \label{background}
Give a good technical overview of your topic with literature references.

\newpage

\section{Process model} \label{process-model}
`Derive' the process model. Include all the necessary data needed. Provide diagrams, etc.

Be as technical as possible, and include a solid qualitative analysis. Include and justify any assumptions.

\newpage

\section{MCNP model} \label{mcnp-model}
Provide the MCNP model Screenshot the configuration in the visual editor. Explain what you are modeling and why it is important. Include the input file in an appendix.

\newpage

\section{Results and discussion} \label{results-discussion}
Present and analyze results.

\newpage

\section{Cross cutting discussions} \label{cross-cutting}
Briefly discuss cross cutting issues related to the project.

\newpage

\section{Future work} \label{future-work}
Outside the scope of the project, what additional work could be done using the existing model? What enhancements can be made to the model and for what purpose(s)?

\newpage

\section{Lessons learned} \label{lessons-learned}

What you personally learned over the course of the project

\newpage

\section{Summary remarks} \label{summary-remarks}

%uncomment to prepare references
%\bibliographystyle{}
%\setlength{\bibhang}{0pt}
%\bibliography{}

\newpage

\noindent\Large{\textbf{Appendix I}} \label{appendix-i}

\newpage

\noindent\textbf{\Large{Tables}}

\begin{table}[h!]
    \centering
    \caption{Caption}
    \label{tab-label}
    \begin{tabular}{|c|c|c|}
        \hline
        A&
        B&
        C\\
        \hline
        1&2&3\\
        \hline
        4&5&6\\
        \hline
        7&8&9\\
        \hline
        X&Y&Z\\
        \hline
    \end{tabular}
\end{table}

\newpage

\noindent\textbf{\Large{Figures}}

%\begin{figure}[h!]
%    \centering
%    \includegraphics[width=0.8\textwidth]{}
%    \caption{}
%    \label{fig-label}
%\end{figure}

\end{document}
