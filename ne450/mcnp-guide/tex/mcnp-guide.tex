\documentclass[11pt,a4paper]{article}
\usepackage[lmargin=1in,rmargin=1in,tmargin=1in,bmargin=1in]{geometry}
\usepackage[pagewise]{lineno} %line numbering
\usepackage{setspace}
\usepackage{ulem} %strikethrough - do not \sout{\cite{}}
\usepackage{xcolor} %change font color
\usepackage{graphicx}
\usepackage{filecontents}
\usepackage{tablefootnote}
\usepackage{footnotehyper}
%\usepackage{subfig}
\usepackage[yyyymmdd]{datetime} %date format
\renewcommand{\dateseparator}{.}
\graphicspath{{../img/}} %path to graphics
\setcounter{secnumdepth}{5} %set subsection to nth level
\usepackage{caption}
\captionsetup[table]{skip=11pt} %sets a space after table caption
\usepackage{times}
\usepackage{tabto} %general tabbed spacing
\usepackage{longtable} %need to put label at top under caption then \\ - use spacing
\usepackage[stable,hang,flushmargin]{footmisc} %footnotes in section titles and no indent
\usepackage[round]{natbib} %parenthesis instead of brackets for inline citations
\usepackage{enumitem}
\usepackage{boldline}
\usepackage{makecell}
\usepackage{booktabs}
\usepackage{amssymb}
\usepackage{amsmath}
\usepackage{physics}
\usepackage{tabularx}
\usepackage{multirow}
\usepackage{lscape}
\usepackage{array}
\usepackage{caption}
\usepackage{subcaption}
\usepackage[labelfont=bf]{caption}
\usepackage{chngcntr}

%\counterwithin{table}{section}

%\usepackage{xr}
%\externaldocument{} %aux file needed

\newcommand{\edit}[1]{\textcolor{blue}{#1}} %shortcut for changing font color on revised text
\newcommand{\fn}[1]{\footnote{#1}} %shortcut for footnote tag

\newcommand*\sq{\mathbin{\vcenter{\hbox{\rule{.3ex}{.3ex}}}}} %makes a small square as a separator $\sq$

\usepackage{fancyhdr}
\pagestyle{fancy}
\fancyhf{} %move page number to bottom right
\renewcommand{\headrulewidth}{0pt} %turn off line in header
%\lhead{\scriptsize Journal}
%\chead{\scriptsize Draft Manuscript - \textit{Christensen and Borrelli}}
\rhead{\scriptsize \today}
\rfoot{\thepage}

\begin{document}

\begin{titlepage}
    \title{MCNP for engineers - A walkthrough on how to use it, get results, and what it all means to a fulfilling life}
    \author{
        Prof. R. A. Borrelli
        \\ \\ \\
        University of Idaho $\sq$ Idaho Falls Center for Higher Education\\
        Center for Advanced Energy Studies\\[0.05in]
        Engineering/Technology Management, Industrial Technology\\and\\Nuclear Engineering Department
        \\ \\ \\
        rborrelli@uidaho.edu
    }
\clearpage %not have page number on title page
\maketitle
\thispagestyle{empty} %start with page number 1 on second page
\end{titlepage}

\noindent\textbf{\Large{Preface}}\\
\noindent\textbf{Who is this guide for?}\\
\noindent Advanced undergraduate students and graduate students in any nuclear engineering curriculum. Students should know - 
\begin{itemize}[topsep=0pt,itemsep=-1ex,partopsep=1ex,parsep=1ex]
    \item Basic nuclear physics; e.g., cross sections
    \item Interactions of neutrons and photons with matter
    \item Shielding dose rates
    \item Four/six factor formula; e.g., what $k_{EFF}$ is,
    \item How a nuclear reactor works
    \item Solving for buckling
    \item Neutron diffusion
\end{enumerate}

\noindent Just about every nuclear engineering department has this course. Typically, the Larmarsh or Duderstadt textbooks are used mostly. Sometimes the Shultis textbook is used. It's usually one of the first classes taken prior to the higher-level nuclear engineering courses. However, this course provides all you need to know to run and understand MCNP. I happen to teach this course. I use Lamarsh with Shultis as a reference. I happend to use Lamarsh because that was the textbook when I first took this kind of course.\\

\noindent\textbf{What is MCNP?}\\
\noindent Your best friend. Your greatest nemesis. MCNP is a contradiction. It will make you suffer, but it will open doors and present new opportunities. \\

\noindent MCNP is a computational tool - that means you're not coding per se. You set up the input file that the code will read and then execute. MCNP tracks neutrons and photons for specified geometries and produces a wealth of resulting data. That seems simple. It's not. With a little guidance, effective modeling with MCNP is achievable.\\ \\

\noindent\textbf{Why is MCNP so important?}
\noindent It's not that necessarily MCNP itself is so important. Neutronics modeling is. We can't design any reactor without knowing where the neutrons are going and what they're going to do when they get there. MCNP happens to be the first neutronics computational tool. (They literally used punch cards.) All other tools are benchmarked against MCNP.\\

\noindent Not everyone is going to be a neutronics expert. For those that want to be, mastering MCNP makes it far easier to learn other neutronics codes, like Serpent. Any facility with MCNP provides a fundamental basis for a career in nuclear engineering. Frankly, if you're a graduate student, you're looking for internships. You may not really care what you're going to do; you just want a good position for your CV and a chance to network for future career development. Absolutely nothing wrong with that. On more than one occasion, I have had a researcher come up to me and ask `Do you know any of the students that know MCNP? I have some money for an intern this summer, but I need someone to step in and get right going.'\\

\noindent Learning MCNP will also lend to transferrable skills. Whether it is good coding practices, geometric modeling, or just developing engineering judgement, this will lead to success in higher endeavors. \\



\end{document}
