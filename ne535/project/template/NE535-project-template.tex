\documentclass[11pt,a4paper]{article}
\usepackage[lmargin=1in,rmargin=1in,tmargin=1in,bmargin=1in]{geometry}
\usepackage[pagewise]{lineno} %line numbering
\usepackage{setspace}
\usepackage{ulem} %strikethrough
\usepackage{xcolor} %change font color
\usepackage{graphicx}
\usepackage{filecontents}
\usepackage{tablefootnote}
\usepackage{subfig}
\usepackage[yyyymmdd]{datetime}
\renewcommand{\dateseparator}{.}
\graphicspath{{../img/}}
\setcounter{secnumdepth}{5} %set subsection to nth level
\usepackage{caption}
\captionsetup[table]{skip=11pt} %sets a space after table caption
\usepackage{times}
\usepackage{enumitem}

\usepackage[round]{natbib} %parenthesis instead of brackets for inline citations

%%%move page number to bottom right
\usepackage{fancyhdr}
\fancyhf{}
\renewcommand{\headrulewidth}{0pt}
\rfoot{\thepage}
\pagestyle{fancy}
%%%

% move page number to bottom right
\usepackage{fancyhdr}
\fancyhf{}
\renewcommand{\headrulewidth}{0pt}
\rfoot{\thepage}
\pagestyle{fancy}

\begin{document}

\begin{titlepage}
    \title{Title}
    \author{
        Name \\
        NE535: Criticality Safety I Course Project \\
        Nuclear Engineering Program, University of Idaho-Idaho Falls \\
        email 
    }
\clearpage % not have page number on title page
\maketitle
\thispagestyle{empty} % start with page number 1 on second page
\end{titlepage}

\onehalfspacing
\linenumbers
\pagewiselinenumbers
\modulolinenumbers[3] % line numbering interval

\noindent\Large{\textbf{Executive Summary}} \\

\newpage

\section{Introduction}
What this CSE will be used for

\newpage

\section{Background}
\subsection{Summarize the facility, equipment, process, system boundaries and interfaces}

\subsection{Define material inputs and outputs}

\subsection{Assumptions and special considerations}

\newpage

\section{Methodology}
\subsection{Methods to evaluate conditions}

\subsection{Calculations}

\subsection{Parameters}
MAGICMERV

Provide enough information so that the reader can follow and replicate

\newpage

\section{Normal operations}
Describe processes in enough detail so that all relevant information is presented to justify the safety of the system

\subsection{Location}

\subsection{Specific equipment}

\subsection{Calculations}

\subsection{Interactions with other systems}

\newpage

\section{Abnormal conditions}

\subsection{Assessment}

\subsection{Accidents}

\subsection{Simulation results}

\subsection{Discussion}

\newpage

\section{Controls and operating limits}

Specific set for each initiating event

Tie controls back to accidents

Analyze response to the accidents

\newpage

\section{Summary remarks}

\newpage

\section{Lessons learned}

What you personally learned over the course of the project

% uncomment to prepare references
%\bibliographystyle{}
%\setlength{\bibhang}{0pt}
%\bibliography{}

\newpage

\noindent\Large{\textbf{Appendix I}}

\end{document}
